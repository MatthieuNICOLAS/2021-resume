%!TEX TS-program = xelatex
%!TEX encoding = UTF-8 Unicode
% Awesome CV LaTeX Template for CV/Resume
%
% This template has been downloaded from:
% https://github.com/posquit0/Awesome-CV
%
% Original author:
% Claud D. Park <posquit0.bj@gmail.com>
% http://www.posquit0.com
%
% Modifications by:
% Junhao Dong <junhao.dong96@gmail.com>
%
% Template license:
% CC BY-SA 4.0 (https://creativecommons.org/licenses/by-sa/4.0/)
%


%-------------------------------------------------------------------------------
% CONFIGURATIONS
%-------------------------------------------------------------------------------
% A4 paper size by default, use 'letterpaper' for US letter
\documentclass[12pt, a4paper]{awesome-cv}

% Configure page margins with geometry
\geometry{left=1.4cm, top=.8cm, right=1.4cm, bottom=1.8cm, footskip=.5cm}

% Specify the location of the included fonts
\fontdir[fonts/]

% Color for highlights
% Awesome Colors: awesome-emerald, awesome-skyblue, awesome-red, awesome-pink, awesome-orange
%                 awesome-nephritis, awesome-concrete, awesome-darknight
\colorlet{awesome}{awesome-red}
% Uncomment if you would like to specify your own color
% \definecolor{awesome}{HTML}{CA63A8}

\usepackage[backend=biber,defernumbers=true,style=numeric,sorting=ydnt,maxbibnames=3]{biblatex}
\addbibresource{references.bib}

\usepackage{acronym} % \ac[p], \acl[p], \acs[p], \acf[p]

\newcommand{\customlink}[2]{
  \href{#2}{\textbf{#1} (\emph{#2})}
}

% Acronyms
% --------
\acrodef{CRDT}[CRDT]{Conflict-free Replicated Data Type}
\acrodefplural{CRDT}[CRDTs]{Conflict-free Replicated Data Types}

% Colors for text
% Uncomment if you would like to specify your own color
% \definecolor{darktext}{HTML}{414141}
% \definecolor{text}{HTML}{333333}
% \definecolor{graytext}{HTML}{5D5D5D}
% \definecolor{lighttext}{HTML}{999999}

% Set false if you don't want to highlight section with awesome color
\setbool{acvSectionColorHighlight}{false}

% If you would like to change the social information separator from a pipe (|) to something else
\renewcommand{\acvHeaderSocialSep}{\quad\textbar\quad}

\makeatletter
\patchcmd{\@sectioncolor}{\color}{\mdseries\color}{}{}
\makeatother

%-------------------------------------------------------------------------------
%	PERSONAL INFORMATION
%	Comment any of the lines below if they are not required
%-------------------------------------------------------------------------------
% Available options: circle|rectangle,edge/noedge,left/right
% \photo[rectangle,edge,right]{profile}
\name{Matthieu}{Nicolas}
\position{Doctorant - Attaché temporaire d'enseignement et de Recherche}
% \address{address}

\mobile{+33 6 75 98 34 40}
\email{matthieu.nicolas@univ-lorraine.fr}
% \homepage{homepage}
\github{MatthieuNICOLAS}
%\linkedin{linkedin-id}
% \gitlab{gitlab-id}
% \stackoverflow{SO-id}{SO-name}
% \twitter{@twit}
% \skype{skype-id}
% \reddit{reddit-id}
% \extrainfo{extra informations}

%-------------------------------------------------------------------------------
\begin{document}

% Print the header with above personal informations
% Give optional argument to change alignment(C: center, L: left, R: right)
\makecvheader[C]

% Print the footer with 3 arguments(<left>, <center>, <right>)
% Leave any of these blank if they are not needed
\makecvfooter
  {\today}
  {Matthieu Nicolas~~~·~~~CV}
  {\thepage}

%-------------------------------------------------------------------------------
%	CV/RESUME CONTENT
%	Each section is imported separately, open each file in turn to modify content
%-------------------------------------------------------------------------------
\cvsection{Déroulement de carrière}

\begin{cventries}

  \cventry
  {Doctorant ATER}
  {Université de Lorraine, Polytech Nancy}
  {Nancy}
  {Depuis Septembre 2020}
  {}

\cventry
  {Doctorant contractuel}
  {Université de Lorraine, Loria, équipe Coast}
  {Nancy}
  {Octobre 2017 - Août 2020}
  {
    \begin{cvitems} % Description(s) of tasks/responsibilities
      \item {\textbf{Intitulé :} (Ré)Identification efficace dans les types de données répliquées sans conflit (CRDTs)}
      \item{\textbf{Mots clés :} systèmes distribués, pair-à-pair, réplication optimiste, CRDTs, performances}
      \item{\textbf{Directeur de thèse :} Pr. Olivier Perrin}
      \item{\textbf{Co-directeur de thèse :} Dr. Gérald Oster}
    \end{cvitems}
  }

\begin{cvparagraph}
  Dans le cadre de mes travaux de recherche, j'étudie et travaille sur les \acfp{CRDT}.
  Les \acp{CRDT} sont de nouvelles spécifications des types abstraits de données, tels que l'\emph{Ensemble} ou la \emph{Séquence}.
  Contrairement aux spécifications traditionnelles, les \acp{CRDT} sont conçus pour supporter nativement les modifications concurrentes.
  Pour ce faire, ces structures de données intègrent un mécanisme de résolution de conflits directement au sein de leur spécification.
  Cette spécificité rend les \acp{CRDT} particulièrement adaptés pour concevoir des systèmes distribués hautement disponibles dans lesquels différents noeuds répliquent et modifient une même donnée sans aucune coordination.

  Pour résoudre les conflits de manière déterministe, les \acp{CRDT} utilisent généralement des identifiants qu'ils associent aux éléments stockés au sein de la structure de données.
  Cependant, selon le type de \ac{CRDT}, les identifiants doivent respecter un ensemble de contraintes telles qu'être unique ou appartenir à un espace dense.
  Dans certains cas, ces contraintes empêchent de borner la taille des identifiants.
  La taille des identifiants croît alors continuellement avec le nombre de modifications effectuées.

  Ces identifiants représentent donc un surcoût lié à l'utilisation des \acp{CRDT} par rapport aux structures de données traditionnelles.
  Ce surcoût décourage l'adoption des \acp{CRDT} dans les systèmes distribués.
  Le but de cette thèse est de proposer des solutions pour pallier ce problème.
  L'approche que nous proposons consiste à intégrer un mécanisme de renommage au sein des \acp{CRDT}.
  Ce mécanisme a pour but de permettre aux différents noeuds de renommer les identifiants afin de réduire leur taille, tout en respectant les contraintes imposées aux \acp{CRDT}.
  En particulier, le renommage doit se faire sans aucune coordination entre les noeuds.
  Afin de valider cette approche, nous avons conçu, implémenté, et évalué un tel mécanisme pour \emph{LogootSplit}, un \ac{CRDT} souffrant particulièrement du problème de croissance des identifiants.
  Ces travaux ont conduit à la conception d'un nouveau \ac{CRDT}, \emph{RenamableLogootSplit}, que j'ai présenté au cours du workshop PaPoC'20.

  \subentrytitlestyle{Publications}
  \begin{description}[labelindent=1.6em,itemsep=-0.3em]
    \item \fullcite{nicolas:hal-02526724}
    \item \fullcite{nicolas:hal-01932552}
  \end{description}
\end{cvparagraph}

\cventry
  {Ingénieur Recherche \& Développement} % Job title
  {INRIA, équipe Coast} % Organization
  {Nancy} % Location
  {Septembre 2014 – Septembre 2017} % Date(s)
  {
    \begin{cvitems} % Description(s) of tasks/responsibilities
      \item {\textbf{Projets :} OpenPaaS::NG et PLM}
    \end{cvitems}
  }

  \honordatestyle{OpenPaaS::NG}
  % 0 OpenPaaS::NG
  % 1 \honortitlestyle{OpenPaaS::NG}\\
  % 2 \honorpositionstyle{OpenPaaS::NG}\\
  % 3 \honorlocationstyle{OpenPaaS::NG}\\
  % 4 \honordatestyle{OpenPaaS::NG}\\
  % 5 \entrytitlestyle{OpenPaaS::NG}\\
  % 6 \subentrytitlestyle{OpenPaaS::NG}\\
  % 7 \entrypositionstyle{OpenPaaS::NG}\\
  % 8 \subentrypositionstyle{OpenPaaS::NG}\\

  \begin{cvparagraph}
    Ce projet avait pour objectif la réalisation d'un réseau social d'entreprise open-source incorporant une suite d'applications collaboratives pair-à-pair de bureautique.
    Le but était ainsi de proposer une alternative viable et libre à des solutions telles que Google Apps. Ce projet fut réalisé en collaboration avec l’équipe DaSciM (Data Science and Mining) du laboratoire d’informatique de l’École Polytechnique, Linagora, XWiki SAS et Nexedi.

    Dans le cadre de ce projet, l'équipe COAST travaillait sur la fédération interorganisationelle de systèmes pair-à-pair et sur la sécurisation des échanges de données dans ce type de collaboration. De plus, elle apportait son expertise sur les mécanismes de réplication de données et de cohérence à terme dans les systèmes distribués.

    C'est sur ce dernier point que portaient les tâches que j'ai effectuées dans le cadre de ce projet.
    Afin d'être validés, ces travaux ont été intégrés dans \customlink{MUTE}{https://www.coedit.re}, la plateforme de démonstration de l'équipe.

    \begin{tightemize}
      \item Maintenance de l'implémentation de \emph{LogootSplit}
      \item Étude de la littérature sur les types de données répliquées sans conflits existants et leurs cas d'utilisation
      \item Développement et intégration d'un système d'anti-entropie
    \end{tightemize}

    \subentrytitlestyle{Publications}
    \begin{description}[labelindent=1.6em,itemsep=-0.3em]
      \item \fullcite{nicolas:hal-01655438}
    \end{description}
  \end{cvparagraph}

  \honordatestyle{ADT PLM}
  % 0 OpenPaaS::NG
  % 1 \honortitlestyle{OpenPaaS::NG}\\
  % 2 \honorpositionstyle{OpenPaaS::NG}\\
  % 3 \honorlocationstyle{OpenPaaS::NG}\\
  % 4 \honordatestyle{OpenPaaS::NG}\\
  % 5 \entrytitlestyle{OpenPaaS::NG}\\
  % 6 \subentrytitlestyle{OpenPaaS::NG}\\
  % 7 \entrypositionstyle{OpenPaaS::NG}\\
  % 8 \subentrypositionstyle{OpenPaaS::NG}\\

  \begin{cvparagraph}
    La \customlink{PLM}{http://people.irisa.fr/Martin.Quinson/Teaching/PLM/} est un environnement d’apprentissage de la programmation libre et ouvert.
    Développé par Gérald Oster et Martin Quinson, il permet d’explorer différents aspects de l’algorithmique au travers d’exercices interactifs.

    Le but de ce projet était de faire évoluer cet outil en une plateforme expérimentale pour l'enseignement de la programmation informatique.
    Pour cela, un mécanisme de capture des traces d'utilisation des apprenant devait être intégré afin de générer un corpus de données.
    Ce corpus, mis à disposition de chercheurs, devait permettre la conduite de travaux de recherche, tel que la conception d'outils d'aide automatique à l'apprentissage.
    Un second objectif ce projet était d'effectuer le portage de l'outil, jusqu'à alors disponible sous la forme d'une application lourde Java, en une application web afin de le rendre accessible au plus grand nombre.

    Mes travaux se sont principalement focalisés sur la réalisation de ce portage.
    Ce changement important de type d'application a entraîné l'apparition de plusieurs problématiques auxquelles il a fallu apporter des solutions.

    \begin{tightemize}
    \item Implémentation et test du mécanisme de capture des traces d'utilisation
    \item Conception et mise en place d'une architecture distribuée assurant le passage à l'échelle de l'application
    \item Isolation de l'exécution du code des apprenants
    \item Déploiement et supervision d'une application multi-composants
    \end{tightemize}
  \end{cvparagraph}

\cventry
  {Stage élève-ingénieur} % Job title
  {Université de Lorraine, équipe Coast} % Organization
  {Nancy} % Location
  {Avril 2014 – Août 2014} % Date(s)
  {
    \begin{cvitems} % Description(s) of tasks/responsibilities
      \item {\textbf{Intitulé :}  Réalisation d’une plateforme d’edition collaborative}
    \end{cvitems}
  }

  \begin{cvparagraph}
    Issue des travaux sur l'édition collaborative, une nouvelle famille d'algorithmes de réplication des données et de maintien de la cohérence à terme est apparue récemment : l'approche \acf{CRDT}.
    Cette nouvelle famille d'algorithme répond à plusieurs des limites constatées chez les autres approches existantes, notamment concernant la capacité de passage à l'échelle.

    L'équipe Coast, travaillant sur ce domaine de recherche, a proposé un nouvel algorithme de cette famille : \emph{LogootSplit}.

    Afin d'illustrer et de mettre en valeur les travaux de l'équipe sur cette approche,
    ma tâche a été de concevoir et de développer un éditeur collaboratif temps réel se basant sur cet algorithme.

    \begin{tightemize}
      \item Implémentation sous forme de librairie de \emph{LogootSplit}
      \item Conception et développement de \customlink{MUTE}{https://www.coedit.re}, un éditeur collaboratif temps réel en ligne reposant sur cette librairie
    \end{tightemize}
  \end{cvparagraph}

\cventry
  {Stage élève-technicien} % Job title
  {École Polytechnique de Montréal} % Organization
  {Montréal, Canada} % Location
  {Avril 2011 – Juin 2011} % Date(s)
  {
    \begin{cvitems} % Description(s) of tasks/responsibilities
      \item {\textbf{Intitulé :}  Développement d’un outil d'analyse d'algorithmes d'édition collaborative}
    \end{cvitems}
  }

\begin{cvparagraph}
  Les outils d'édition collaborative existants reposent majoritairement sur une famille spécifique d'algorithmes pour assurer le maintien de la cohérence à terme : les transformées opérationnelles.

  Deux propriétés de convergence \emph{TP1} et \emph{TP2} existent et permettent de garantir la correction de ces algorithmes.

  L'objectif de ce stage était de réaliser un outil permettant de vérifier automatiquement le respect de ces propriétés pour un algorithme donné.
  \begin{tightemize}
    \item Implémentation de plusieurs algorithmes issus de la famille des transformées opérationnelles
    \item Développement de l'outil permettant de vérifier les propriétés de convergences \emph{TP1} et \emph{TP2} pour les algorithmes implémentés.
  \end{tightemize}
\end{cvparagraph}

\end{cventries}

\cvsection{Diplômes}

\begin{cventries}

\cventry
    {Diplôme d'ingénieur TELECOM Nancy, spécialité Ingénierie du Logiciel} % Job title
    {TELECOM Nancy} % Organization
    {Nancy} % Location
    {2011 – 2014} % Date(s)
    {}

\cventry
  {Diplôme Universitaire de Technologie, spécialité Informatique} % Job title
  {IUT de Metz} % Organization
  {Metz} % Location
  {2009 – 2011} % Date(s)
  {}
\end{cventries}


\cvsection{Enseignement}

\begin{cvparagraph}
  J'ai effectué un service de demi-ATER cette année à Polytech Nancy.
  Auparavant, j'avais assuré une charge d'enseignement en tant que DCCE à l'IUT Nancy Charlemagne et effectué plusieurs vacations à TELECOM Nancy ainsi qu'à la Faculté des Sciences et Technologies de Nancy.
  L'ensemble de mes activités d'enseignement est présenté ci-dessous et récapitulé dans \autoref{tab:enseignement-details}.
  Au total, je comptabilise \textbf{350h équivalent TD} d'enseignement.
\end{cvparagraph}

\begin{table}[ht]
  \centering
  \begin{tabular}{ c | c | c | c c c c | c }
    & Licence 1 & Licence 2 & \multicolumn{4}{c|}{Licence 3} \\
    Année universitaire & TD & TP & EI & CM & TD & TP & eq. TD  \\
    \hline
    2014 -- 2015 & - & - & - & - & - & 30~h & 20~h\\
    2015 -- 2016 & - & - & - & - & - & 30~h & 20~h\\
    2016 -- 2017 & - & 18~h & - & - & - & 30~h & 32~h\\
    2017 -- 2018 & 42~h & - & - & - & - & 24~h & 58~h\\
    2018 -- 2019 & 32~h & - & 24~h & - & - & - & 62~h\\
    2019 -- 2020 & 32~h & - & 24~h & - & - & - & 62~h\\
    2020 -- 2021 & 44~h & - & - & 12~h & 34~h & - & 96~h\\
    \hline
    Total  & 150~h & 18~h & 48~h & 12~h & 34~h & 114~h & 350~h\\
  \end{tabular}
  \caption{Volume horaire d'enseignement par type, niveau, et année universitaire}\label{tab:enseignement-details}
\end{table}

\begin{cventries}
  \cventry
  {Projet Informatique}
  {Polytech Nancy}
  {}
  {2020 - 2021}
  {
    \begin{cvitems}
      \item {\textbf{Niveau : } Polytech Nancy 3A spécialité IA2R (Licence 3)}
      \item {\textbf{Responsable : } Dr. Vincent Despré}
      \item {\textbf{Volume horaire : } 10h TD}
    \end{cvitems}
  }

  \begin{cvparagraph}
    L'objectif de ce module est de renforcer et valider les compétences acquises par les étudiants via la conception et réalisation d'une application. Ce projet étant de plus grande envergure que ceux réalisés jusqu'alors, ce module insiste sur les bonnes pratiques de la conception orientée objet (séparation des responsabilités, factorisation du code, minimisation des dépendances...) et de développement (logiciels de gestion de versions, gestionnaires de dépendances...).
  \end{cvparagraph}

  \begin{cvparagraph}
    Chargé de TD, j'ai encadré des groupes d'étudiants dans la conception et réalisation de l'application demandée. J'ai aussi préparé et présenté en introduction de ce module un cours portant sur la conception orientée objet.
  \end{cvparagraph}

  \cventry
  {Base de Données II}
  {Polytech Nancy}
  {}
  {2020 - 2021}
  {
    \begin{cvitems}
      \item {\textbf{Niveau : } Polytech Nancy 3A spécialité IA2R (Licence 3)}
      \item {\textbf{Responsable : } Pr. Claude Godart}
      \item {\textbf{Volume horaire : } 6h CM et 12h TD}
    \end{cvitems}
  }

  \begin{cvparagraph}
    Suite directe du module de Bases de Données I, ce module enseigne aux étudiants les bases du langage SQL, comment déployer et utiliser une base de données à l'aide d'un SGBD Relationnel et finalement comment interagir avec une base de données depuis un langage de programmation.
  \end{cvparagraph}

  \begin{cvparagraph}
    Chargé de CM et de TD, j'ai actualisé les différents supports de cours et sujets de TD fournis.
  \end{cvparagraph}

  \cventry
  {Base de Données I}
  {Polytech Nancy}
  {}
  {2020 - 2021}
  {
    \begin{cvitems}
      \item {\textbf{Niveau : } Polytech Nancy 3A spécialité IA2R (Licence 3)}
      \item {\textbf{Responsable : } Pr. Claude Godart}
      \item {\textbf{Volume horaire : } 6h CM et 12h TD}
    \end{cvitems}
  }

  \begin{cvparagraph}
    L'objectif de ce module est de présenter aux étudiants le concept de Base de Données et de leur enseigner comment en concevoir et en utiliser une. Ce module se focalise sur la réalisation du schéma conceptuel d'un système d'information, la méthode pour concevoir un schéma relationnel équivalent, la normalisation de ce dernier et finalement son utilisation à l'aide de l'algèbre relationnel.
  \end{cvparagraph}

  \begin{cvparagraph}
    Chargé de CM et de TD, j'ai actualisé les différents supports de cours et sujets de TD fournis.
  \end{cvparagraph}

  \cventry
  {Découverte de l'informatique}
  {Polytech Nancy}
  {}
  {2020 - 2021}
  {
    \begin{cvitems}
      \item {\textbf{Niveau : } Polytech Nancy 1A (Licence 1)}
      \item {\textbf{Responsable : } Charles Dumenil}
      \item {\textbf{Volume horaire : } 44h TD}
    \end{cvitems}
  }

  \begin{cvparagraph}
    Le but de ce module est d'initier les étudiants à la programmation informatique. Ce module présente les bases de l'algorithmie (instructions conditionnelles, boucles, fonctions...) et met en application ces notions dans le cadre d'exercices à réaliser en Python.
  \end{cvparagraph}

  \begin{cvparagraph}
    Chargé de TD.
  \end{cvparagraph}

  \cventry
  {Programmation web sur client} % Job title
  {IUT Nancy Charlemagne} % Organization
  {} % Location
  {2017 - 2020} % Date(s)
  {
    \begin{cvitems}
      \item {\textbf{Niveau : } Licence Pro Informatique CIASIE}
      \item {\textbf{Responsable : } Dr. Gérôme Canals}
      \item {\textbf{Volume horaire : } 24h TP (2017-2018) puis 24h EI (2018-2020)}
    \end{cvitems}
  }

  \begin{cvparagraph}
    Destiné à des étudiants ayant déjà appris et utilisé JavaScript au cours de leur formation précédente, ce module a pour but de consolider leur connaissance et maîtrise des bases du langage (POO, manipulation du DOM...) puis d'introduire des notions plus avancées (closures, AJAX, bundling...).
  \end{cvparagraph}

  \begin{cvparagraph}
    Chargé de CM et TP (Enseignement Intégré), j'ai notamment retravaillé le contenu du module (cours, exercices, projet) par rapport à l'évolution du langage à partir de la seconde année.
  \end{cvparagraph}

  \cventry
  {Algorithmique} % Job title
  {IUT Nancy Charlemagne} % Organization
  {} % Location
  {2018 - 2020} % Date(s)
  {
    \begin{cvitems}
      \item {\textbf{Niveau : } DUT Informatique 1A}
      \item {\textbf{Responsable : } Dr. Yolande Belaïd}
      \item {\textbf{Volume horaire : } 32h TD}
    \end{cvitems}
  }

  \begin{cvparagraph}
    Destiné à des étudiants débutant leurs études en informatique, l'objectif de ce module est de leur présenter la notion d'algorithme et de leur enseigner comment en concevoir (instructions disponibles, décomposition de problèmes en sous-problèmes...).
  \end{cvparagraph}

  \begin{cvparagraph}
    Chargé de CM et TD (Enseignement Intégré).
  \end{cvparagraph}

  \cventry
  {Conception Orientée Objet} % Job title
  {IUT Nancy Charlemagne} % Organization
  {} % Location
  {2017 - 2018} % Date(s)
  {
    \begin{cvitems}
      \item {\textbf{Niveau : } DUT Informatique 1A}
      \item {\textbf{Responsable : } Dr. Vincent Thomas}
      \item {\textbf{Volume horaire : } 42h TD}
    \end{cvitems}
  }

  \begin{cvparagraph}
    L'objectif de ce module est d'enseigner aux étudiants les principes de la conception orientée objet (séparation des responsabilités, factorisation du code...) et de leur apprendre à manier les outils existants (UML, patrons de conceptions...).
    Une partie du module est aussi consacrée aux bonnes pratiques de développement (logiciels de gestion de versions, tests unitaires...).
  \end{cvparagraph}

  \begin{cvparagraph}
    Chargé de TD et TP.
  \end{cvparagraph}

  \cventry
  {Bases de la Programmation Objet} % Job title
  {Faculté des Sciences et Technologies de Nancy} % Organization
  {} % Location
  {2016 - 2017} % Date(s)
  {
    \begin{cvitems}
      \item {\textbf{Niveau : } Licence 2 Informatique}
      \item {\textbf{Responsable : } Dr. Martine Gautier}
      \item {\textbf{Volume horaire : } 18h TP}
    \end{cvitems}
  }

  \begin{cvparagraph}
    L'objectif de ce module est d'enseigner aux étudiants le paradigme de la programmation orientée objet et ses spécificités (classe, héritage, polymorphisme...), ainsi que les bonnes pratiques de développement (programmation par contrat, tests...).
    Ces concepts sont ensuite mis en application dans le cadre de multiples exercices à réaliser en Java.
  \end{cvparagraph}

  \begin{cvparagraph}
    Chargé de TP, j'ai notamment participé à la conception et l'animation du TP noté.
  \end{cvparagraph}

  \cventry
  {Préparation informatique} % Job title
  {TELECOM Nancy} % Organization
  {} % Location
  {2014 - 2017} % Date(s)
  {
    \begin{cvitems}
      \item {\textbf{Niveau : } TELECOM Nancy 1A (Licence 3)}
      \item {\textbf{Responsables : } Dr. Gérald Oster et Pr. Martin Quinson}
      \item {\textbf{Volume horaire : } 30h TP}
    \end{cvitems}
  }

  \begin{cvparagraph}
    Destiné aux élèves provenant de classes préparatoires, ce module a pour but de travailler les notions de bases de la programmation (instructions, conditions, boucles...) avant d'aborder des exercices plus complexes (tris, recursivité).
    Dans le cadre de ce module, les étudiants travaillent de façon autonome sur l'environnement d'apprentissage de la PLM.
  \end{cvparagraph}

  \begin{cvparagraph}
    Chargé de TP.
  \end{cvparagraph}

\end{cventries}

\cvsection{Encadrement}

\begin{cventries}

\cventry
  {Stage TELECOM Nancy 2a}
  {~}
  {}
  {}
  {
    \vspace{-5mm}
    \begin{cvitems}
      \item Ishara Chan-Tung : \emph{Intégration d'un agent de messages basé sur des journaux au sein d'une application d'édition collaborative}, juin à août 2019, co-encadrement avec Cédric Enclos
      \item Pierric Grguric : \emph{Service de compilation isolé pour la PLM}, juin à août 2015, co-encadrement avec Gérald Oster et Martin Quinson
      \item Alexandre Carpentier : \emph{Remédiation de masse dans un environnement d'apprentissage}, juin à août 2015, co-encadrement avec Gérald Oster et Martin Quinson
      \item Tanguy Gloaguen : \emph{Mise en place d'un environnement de qualification pour la plateforme PLM}, juin à août 2015, co-encadrement avec Gérald Oster et Martin Quinson
    \end{cvitems}
}

\cventry
  {Projet d'initiation à la recherche TELECOM Nancy 2a}
  {~}
  {}
  {}
  {
    \vspace{-5mm}
    \begin{cvitems}
      \item Pierre Maeckereel, Yannick Philippe : \emph{Simulation du comportement de collaborateurs dans une session d’edition collaborative}, janvier à mai 2017, co-encadrement avec Quentin Laporte-Chabasse
    \end{cvitems}
  }

\cventry
  {Stage DUT}
  {~}
  {}
  {}
  {
    \vspace{-5mm}
    \begin{cvitems}
      \item Tom Mendez-Porcel : \emph{Implémentation d'un protocole de gestion de groupe au sein d'une application d'édition collaborative}, co-encadrement avec Victorien Elvinger, avril à juillet 2020
      \item Théodore Lambolez : \emph{Portage web d'un exerciseur de programmation}, avril à juin 2015, co-encadrement avec Gérald Oster et Martin Quinson
      \item Baptiste Mounier : \emph{Langage visuel pour un exerciseur}, avril à juin 2015, co-encadrement avec Gérald Oster et Martin Quinson
      \item Benjamin Thirion : \emph{Conception et réalisation d'un éditeur d'exercices de programmation}, avril à juin 2015, co-encadrement avec Gérald Oster et Martin Quinson
    \end{cvitems}
  }
\end{cventries}

%-------------------------------------------------------------------------------
\end{document}
